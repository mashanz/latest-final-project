\chapter*{Abstrak}

\noindent \textit{System on a Chip} (SoC) adalah sebuah modul \textit{embedded system} yang
memiliki fungsi tertentu dalam sebuah papan \textit{chip silicon} yang juga bisa disebut
dengan \textit{Veri Large Scale Integration} (VLSI). Pemilik dari desain SoC memiliki
hak cipta atas desain sistem yang telah dibuat. \textit{Fabless} manufacturing merupakan
cara pencetakan modul perangkat keras yang desainer \textit{Integrated Circuit} (IC)
adalah \textit{Outsourching} dari luar pabrik percetakan.

\vspace*{0.5cm}
\noindent \textit{Fabless} manufacturing dari desain IC memiliki celah pencurian desain
ketika desain akan dicetak atau ketika proyek membutuhkan \textit{mutiple module}
dengan berbagai fungsi dari berbagai desainer. Oleh karena itu setiap modul VLSI
dari desainer chip ini membutuhkan bukti \textit{ownership} dari perancang atau
perusahaan produksi. Dalam penelitian ini dibuat verifikasi \textit{ownership}
dengan 2 kunci khusus verifikasi yaitu \textit{Polygate} sebagai kunci utama yang akan
mengaktifkan kunci kedua, dan kunci kedua akan aktif yang prosesnya
menggunakan algoritme filter digital.

\vspace*{0.5cm}
\noindent Pengamanan menggunakan algoritma pengecoh/pembingung (\textit{Obfuscation}) untuk melindungi rangkaian utama. Rangkaian utama disisipkan dengan rangkaian pelindung tanpa merubah dan mengganggu fungsi utama rangkaian. Teknik pengecohan dilakukan pada \textit{behavioral level} dan \textit{sinthesis level}. Pada hasil kompilasi desain sintesis (RTL) didapat rangkaian utama dan pelindung tercampur menjadi satu. Sehingga pada hasil akhir desain seakan tidak ada rangkaian lain selain rangkaian utama. Serta apabila rangkaian berhasil di gandakan (\textit{cloning}) maka rangkaian tersebut dapat diklaim dengan menggunakan alat kusus untuk mengaktifkan rangkaian pelindung.

\vspace*{0.5cm}

\noindent \textbf{Kata Kunci}: VLSI, \textit{Intelectual Property Protection}, \textit{Digital Signal Processing}, \textit{Polygate Watermark}.

\newpage