%%%%%%%%%%%%%%%%%%%%%%%%%%%%%%%%%%%%%%%%%%%%%%%%%%%%%%%%%%%%%
% 
%%%%%%%%%%%%%%%%%%%%%%%%%%%%%%%%%%%%%%%%%%%%%%%%%%%%%%%%%%%%%

\chapter{\kesimpulan}

%%%%%%%%%%%%%%%%%%%%%%%%%%%%%%%%%%%%%%%%%%%%%%%%%%%%%%%%%%%%%
% 
%%%%%%%%%%%%%%%%%%%%%%%%%%%%%%%%%%%%%%%%%%%%%%%%%%%%%%%%%%%%%

\section{Kesimpulan}
Meninjau hasil analisis yang di lakukan dari pengujian yang telah dilakukan, di dapat kesimpulan sebagai berikut.

\begin{enumerate}	
	\item Masih memungkinkan menyisipkan suatu rangkaian pelindung ke dalam rangkaian utama tanpa mengganggu fungsi utama rangkaian.
	
	\item Data signature dapat di panggil, sehingga desain dapat diklaim kepemilikannya. 
	
	\item Terjadi penurunan kecepatan proses serta peningkatan kebutuhan daya pada hasil analisis simulasi di fpga.
	
	\item Dari hasil kompilasi gerbang, komponen komponen yang digunakan untuk fungsi rangkaian utama dan fungsi rangkaian pelindung menyatu, sehingga bisa digunakan sebagai pengecoh agar rangkaian sulit ditiru.
\end{enumerate}

%%%%%%%%%%%%%%%%%%%%%%%%%%%%%%%%%%%%%%%%%%%%%%%%%%%%%%%%%%%%
% 
%%%%%%%%%%%%%%%%%%%%%%%%%%%%%%%%%%%%%%%%%%%%%%%%%%%%%%%%%%%%

\section{Saran}
Agar Teknologi pengamanan Intelektual Properti lebih maju serta memperbaiki permasalahan yang masih ada pada penelitian ini, berikut beberapa saran dari untuk pengembangan dan penelitian selanjutnya:

\begin{enumerate}
	
	\item Penelitian ini dilakukan pada layer software, untuk meningkatkan kecepatan akses dan mengurangi konsumsi daya dari hasil analisis level behavioral pada fpga, dibutuhkan analisis lebih lanjut pada syntesis level gate (netlist) dan level phisical (layout).
	
	\item Saat ini teknologi serta teknik perlindungan properti intelektual perangkat keras masih terbilang baru, bidang keamanan pada IC masih minim resource serta proses manufakturing IC sendiri begitu kompleks dan luas serta spesifikasi desain setiap produk sangat rahasia. Dibutuhkan kajian khusus keamanan pada Level fabrikasi seperti RTL Level, Gate Level dan Layout Level.
\end{enumerate}
