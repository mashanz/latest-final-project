%-----------------------------------------------------------------------------%
\chapter{\babTiga}
%-----------------------------------------------------------------------------%
\todo{tambahkan kata-kata pengantar bab 1 disini}


%-----------------------------------------------------------------------------%
\section{Satu Persamaan}
%-----------------------------------------------------------------------------%

\noindent \begin{align}\label{eq:garis}
	\cfrac{y - y_{1}}{y_{2} - y_{1}} = 
	\cfrac{x - x_{1}}{x_{2} - x_{1}}
\end{align}

\equ~\ref{eq:garis} diatas adalah persamaan garis. 
\equ~\ref{eq:garis} dan \ref{eq:bola} sama-sama dibuat dengan perintah \bslash
align. 
Perintah ini juga dapat digunakan untuk menulis lebih dari satu persamaan. 

\noindent \begin{align}\label{eq:bola}
	\underbrace{|\overline{ab}|}_{\text{pada bola $|\overline{ab}| = r$}} 
		= \sqrt[2]{(x_{b} - x_{a})^{2} + (y_{b} - y_{a})^{2} + 
				\vert\vert(z_{b} - z_{a})^{2}}
\end{align}

%-----------------------------------------------------------------------------%
\section{Lebih dari Satu Persamaan}
\label{sec:multiEqu}
%-----------------------------------------------------------------------------%
\noindent \begin{align}\label{eq:matriks}	
	|\overline{a} * \overline{b}| &= |\overline{a}| |\overline{b}| \sin\theta 
		\\[0.2cm]
	\overline{a} * \overline{b} &=  
		\begin{array}{| c c c |}
			\hat{i} & x_{1} & x_{2} \\
			\hat{j} & y_{1} & y_{2} \\
			\hat{k} & z_{1} & z_{2} \\
		\end{array} \nonumber \\[0.2cm]
	&= \hat{i} \,
		\begin{array}{ | c c | }
			y_{1} & y_{2} \\
			z_{1} & z_{2} \\
		\end{array} 
	   + \hat{j} \,
		\begin{array}{ | c c | }
			z_{1} & z_{2} \\
			x_{1} & x_{2} \\
		\end{array} 
	   + \hat{k} \,	
		\begin{array}{ | c c | }
			x_{1} & x_{2} \\
			y_{1} & y_{2} \\
		\end{array}
		\nonumber
\end{align}

Pada \equ~\ref{eq:matriks} dapat dilihat beberapa baris menjadi satu bagian 
dari \equ~\ref{eq:matriks}. 
Sedangkan dibawah ini dapat dilihat bahwa dengan cara yang sama, \equ~
\ref{eq:gabungan1}, \ref{eq:gabungan2}, dan \ref{eq:gabungan3} memiliki nomor 
persamaannya masing-masing. 

\noindent \begin{align}\label{eq:gabungan1}	
	\int_{a}^{b} f(x)\, dx + \int_{b}^{c} f(x) \, dx = \int_{a}^{c} f(x) \, dx
		\\\label{eq:gabungan2}
	\lim_{x \to \infty} \frac{f(x)}{g(x)} = 0 \hspace{1cm} 
		\text{jika pangkat $f(x)$ $<$ pangkat $g(x)$} \\\label{eq:gabungan3}
	a^{m^{a \, ^{n}\log b }} = b^{\frac{m}{n}}
\end{align}

