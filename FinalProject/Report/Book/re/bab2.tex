%-----------------------------------------------------------------------------%
\chapter{\babDua}
%-----------------------------------------------------------------------------%

%-----------------------------------------------------------------------------%
\section{Pekerjaan Sebelumnya dan Keterkaitan}
%-----------------------------------------------------------------------------%
Secara garis besar teknik Intelectual Property Protection (IPP)
watermarking dapat diklasifikasikan menjadi 2 kelas yaitu Dynamic
Watermarking dan Static Watermarking. Dynamic Watermarking merupakan
watermark yang tidak dapat terdeteksi kecuali dengan menjalankan IP yang telah
di-watermark untuk mendeteksi sinyal yang dihasilkan, seperti digital signal
processing (DSP), atau finite state mechine (FSM) watermarking. Static
Watermarking merupakan watermark yang mengacu pada properti dari sebuah
desain, dan hanya bisa terdeteksi dengan cara statis yang berbeda, seperti jalur dan
penempatan watermarking [7].

Salah satu pengamanan lain adalah mengonversi fail simulasi dari fail.
RTL source code yang memungkinkan tidak mudah untuk di-reverse-engineering
oleh pihak ketiga, sehingga model tidak dapat dirubah dan digunakan kembali
dengan keperluan lain oleh pihak ketiga dan pengguna yang tidak bertanggung
jawab.[8][9]

Namun cara tersebut hanya melindungi dari sisi softwere yang melindungi
IP agar tidak di-salah-gunakan oleh pengguna pihak ketiga. Untuk pengamanan IP
yang digunakan dalam sharing project dan reusable project dapat digunakan
dengan pengamanan Digital Signal Processing cell yang memungkinkan integrasi
dalam sistem.

Dalam penelitian kali ini akan melakukan kombinasi dari proteksi IP
polimorph gate dengan algoritme filter digital. Menggunakan gabungan dari dua
teknik ini akan memberikan tambahan keamanan pada proteksi IP yang
kemungkinan tingkat over write watermark lebih kecil. Oleh karena itu dalam
penelitian ini mengajukan sebuah gabungan metode yang sudah ada untuk
meningkatkan kemampuan pengamanan dalam sebuah modul VLSI yang sudah
ada. Dengan menggabungkan polygate sebagai kunci kombinasi untuk
mengaktifkan modul filter digital yang akan digunakan sebagai watermark.

\section{Perancangan dan Implementasi Algoritme DSP untuk IPP}

Melakukan analisis terhadap masalah yang dikaji kemudian akan
dilakukan rancangan Intelectual Property Protection (IPP) dengan algoritme Filter
Digital yang dibangun meliputi rangkaian uji. Dari desain modul VLSI yang telah
ada akan diuji coba kan performa sebelum diberi watermark.

Dengan memberikan rangkaian watermark sebagai pengamanan pada
blueprint VLSI siap cetak yang menandakan kepemilikan dari desainer atau
perusahaan produsen modul akan melindungi dari kecurangan pihak lain yang
akan mencuri desain tersebut. Sehingga kemungkinan pencurian atau plagiarisme
berkurang yang menyebabkan kerugian pada perusahaan atau desainer karena
desain nya dicuri atau diplagiat.

Desain akan dirancang dengan kombinasi Low Pass Filter, High Pass
Filter, Band Pass Filter, dan Band Reject Filter. Kombinasi ini akan ditentukan
dan diaktifkan oleh polygate sebagai kunci pengaktifan kombinasi Filter digital.
Setelah Filter digital aktif maka kombinasi data akan melewati kombinasi filter
yang diaktifkan dari kombinasi polygate. Kemudian data hasil kombinasi proses
ini akan membentuk pola khusus yang menjadi data watermark dari desainer yang
mencirikan identitas desainer. Setelah diberikan watermark maka modul akan
diuji coba kan kembali performa nya. Bila terjadi penurunan performa maka akan
dilakukan perbaikan algoritma kemudian dilakukan diuji kembali performa nya.
Hingga didapat performa yang paling baik dari beberapa uji coba yang akan
dilakukan.
