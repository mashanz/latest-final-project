\documentclass[12pt, a4paper, onecolumn, oneside, final]{report}

\usepackage{uithesis}

%-----------------------------------------------------------------------------%
% Informasi Mengenai Dokumen
%-----------------------------------------------------------------------------%
% 
\usepackage{booktabs}
\usepackage{bigstrut}

\usepackage{colortbl}
\usepackage{etoolbox}
\patchcmd{\tableofcontents}{\@starttoc}{\vspace{-1cm}\@starttoc}{}{}
%-----------------------------------------------------------------------------%
\setlength{\parindent}{0em}
\setlength{\parskip}{1em}

% Judul laporan. 
\var{\judul}{DESAIN DAN SIMULASI
	PERLINDUNGAN PROPERTI INTELEKTUAL
	MENGGUNAKAN ALGORITME FILTER DIGITAL}
% 
% Tulis kembali judul laporan, kali ini akan diubah menjadi huruf kapital
\Var{\Judul}{DESAIN DAN SIMULASI
	PERLINDUNGAN PROPERTI INTELEKTUAL
	MENGGUNAKAN ALGORITME FILTER DIGITAL}
% 
% Tulis kembali judul laporan namun dengan bahasa Ingris
\var{\judulInggris}{DESIGN AND SIMULATION OF INTELECTUAL PROPERTIES PROTECTION
	USING DIGITAL FILTER ALGORITHM}

% 
% Tipe laporan, dapat berisi Skripsi, Tugas Akhir, Thesis, atau Disertasi
\var{\type}{Tugas Akhir}
% 
% Tulis kembali tipe laporan, kali ini akan diubah menjadi huruf kapital
\Var{\Type}{Tugas Akhir}
% 
% Tulis nama penulis 
\var{\penulis}{Hanjara Cahya Adhyatma}
% 
% Tulis kembali nama penulis, kali ini akan diubah menjadi huruf kapital
\Var{\Penulis}{Hanjara Cahya Adhyatma}
% 
% Tulis NPM penulis
\var{\npm}{1104131113}
% 
% Tuliskan Fakultas dimana penulis berada
\Var{\Fakultas}{Teknik Elektro}
\var{\fakultas}{Teknik Elektro}
% 
% Tuliskan Program Studi yang diambil penulis
\Var{\Program}{S1 Sistem Komputer}
\var{\program}{S1 Sistem Komputer}
% 
% Tuliskan tahun publikasi laporan
\Var{\bulanTahun}{2017}
% 
% Tuliskan gelar yang akan diperoleh dengan menyerahkan laporan ini
\var{\gelar}{S1 Sistem Komputer}
% 
% Tuliskan tanggal pengesahan laporan, waktu dimana laporan diserahkan ke 
% penguji/sekretariat
\var{\tanggalPengesahan}{XX Juli 2017} 
% 
% Tuliskan tanggal keputusan sidang dikeluarkan dan penulis dinyatakan 
% lulus/tidak lulus
\var{\tanggalLulus}{XX Januari 2010}
% 
% Tuliskan pembimbing 
\var{\pembimbing}{Prof. XXXX}
% 
% Alias untuk memudahkan alur penulisan paa saat menulis laporan
\var{\saya}{Penulis}

%-----------------------------------------------------------------------------%
% Judul Setiap Bab
%-----------------------------------------------------------------------------%
% 
% Berikut ada judul-judul setiap bab. 
% Silahkan diubah sesuai dengan kebutuhan. 
% 
\Var{\kataPengantar}{KATA PENGANTAR}
\Var{\babSatu}{Pendahuluan}
\Var{\babDua}{Tinjauan Pustaka}
\Var{\babTiga}{Desain dan Simulasi}
\Var{\babEmpat}{Pengujian dan Analisis}
\Var{\babLima}{Kesimpulan dan Saran}
\Var{\babEnam}{Apa Ya}
\Var{\kesimpulan}{Kesimpulan dan Saran}


\include{hype.indonesia}

\input{istilah}

\begin{document}

\include{sampul}

\pagenumbering{roman}

\addChapter{HALAMAN JUDUL}
\begin{titlepage}
	\begin{center}
		\vspace*{1.0cm}
		% judul thesis harus dalam 14pt Times New Roman
		\bo{\Judul} \\[1.0cm]
		\bo{\textit{\judulInggris}} \\[1.0cm]  
		% harus dalam 14pt Times New Roman
		\bo{\Type} \\[1.0cm]
		% keterangan prasyarat
		\bo{Disusun sebagai syarat untuk memperoleh gelar Sarjana Teknik \\
			pada Program Studi \gelar\\
			Universitas Telkom}\\[1.0cm]
		\bo{Oleh}\\[1.0cm]
		% penulis dan npm
		\bo{\Penulis} \\
		\bo{\npm} \\
		
		\vspace*{1.0cm}
		\begin{figure}
			\begin{center}
				\includegraphics[width=4.5cm]{pics/telu.png}
			\end{center}
		\end{figure}    
		\vspace*{0cm}
		% informasi mengenai fakultas dan program studi
		\bo{
			FAKULTAS \Fakultas\\
			Universitas Telkom \\
			Bandung \\
			\bulanTahun
		}
	\end{center}
\end{titlepage}

\setcounter{page}{2}

\addChapter{LEMBAR PENGESAHAN}
\chapter*{HALAMAN PERSETUJUAN}

\vspace*{0.2cm}

\newpage

\addChapter{LEMBAR PERNYATAAN ORISINALITAS}
\chapter*{\uppercase{halaman pernyataan orisinalitas}}
\vspace*{2cm}

\newpage

\addChapter{ABSTRAK}
\chapter*{Abstrak}

\noindent \textit{System on a Chip} (SoC) adalah sebuah modul \textit{embedded system} yang
memiliki fungsi tertentu dalam sebuah papan \textit{chip silicon} yang juga bisa disebut
dengan \textit{Veri Large Scale Integration} (VLSI). Pemilik dari desain SoC memiliki
hak cipta atas desain sistem yang telah dibuat. \textit{Fabless} manufacturing merupakan
cara pencetakan modul perangkat keras yang desainer \textit{Integrated Circuit} (IC)
adalah \textit{Outsourching} dari luar pabrik percetakan.

\vspace*{0.5cm}
\noindent \textit{Fabless} manufacturing dari desain IC memiliki celah pencurian desain
ketika desain akan dicetak atau ketika proyek membutuhkan \textit{mutiple module}
dengan berbagai fungsi dari berbagai desainer. Oleh karena itu setiap modul VLSI
dari desainer chip ini membutuhkan bukti \textit{ownership} dari perancang atau
perusahaan produksi. Dalam penelitian ini dibuat verifikasi \textit{ownership}
dengan 2 kunci khusus verifikasi yaitu \textit{Polygate} sebagai kunci utama yang akan
mengaktifkan kunci kedua, dan kunci kedua akan aktif yang prosesnya
menggunakan algoritme filter digital.

\vspace*{0.5cm}
\noindent Pengamanan menggunakan algoritma pengecoh/pembingung (\textit{Obfuscation}) untuk melindungi rangkaian utama. Rangkaian utama disisipkan dengan rangkaian pelindung tanpa merubah dan mengganggu fungsi utama rangkaian. Teknik pengecohan dilakukan pada \textit{behavioral level} dan \textit{sinthesis level}. Pada hasil kompilasi desain sintesis (RTL) didapat rangkaian utama dan pelindung tercampur menjadi satu. Sehingga pada hasil akhir desain seakan tidak ada rangkaian lain selain rangkaian utama. Serta apabila rangkaian berhasil di gandakan (\textit{cloning}) maka rangkaian tersebut dapat diklaim dengan menggunakan alat kusus untuk mengaktifkan rangkaian pelindung.

\vspace*{0.5cm}

\noindent \textbf{Kata Kunci}: VLSI, \textit{Intelectual Property Protection}, \textit{Digital Signal Processing}, \textit{Polygate Watermark}.

\newpage

\addChapter{ABSTRACT}
\chapter*{ABSTRACT}

\noindent System on a Chip (SoC) is an embedded system module
Has a certain functionality in a silicon chip board that can also be called
With Veri Large Scale Integration (VLSI). The owner of the SoC design has
Copyright over the system design that has been created. Fabless manufacturing is
How to mold a hardware module that is designer Integrated Circuit (IC)
Is Outsourching from outside the printing factory.

\vspace*{0.5cm}
\noindent Fabless manufacturing from IC design has gap design theft
When the design will be printed or when the project requires mutiple module
With various functions from various designers. Therefore every module is VLSI
From this chip designer requires proof of ownership from the designer or
Production company. In this study writer make a verification of ownership design
with 2 dedicated verification keys ie Polygate as the primary key going
Activate the second key, and the second key will be active which process
Using a digital filter algorithm.

\vspace*{0.5cm}
\noindent Security uses the Obfuscation algorithm to protect the main circuit. The main circuit is inserted with a protective circuit without altering and disrupting the main function of the circuit. The Obfuscation technique is performed on the behavioral level and synthesis level. In the compilation of synthesis design (RTL) obtained main circuit and protector mixed into one. So in the final design as there, it look like is no other circuit other than the main circuit. And if the circuit is successfully cloned then the circuit can be claimed by using a special tool to activate the protective circuit.

\vspace*{0.5cm}

\noindent \textbf{Keywords}: VLSI, Intellectual Property Protection, Digital Signal Processing, Polygate Watermark.

\newpage

\addChapter{\kataPengantar}
\chapter*{\kataPengantar}

Puji syukur terhadap Tuhan Yang Maha Esa yang telah memberikan
rahmat dan hidayah Nya serta nikmat sehat dan nikmat waktu sehingga proposal
ini diselesaikan. Ucapan terima kasih juga diperuntukkan untuk orang tua dan
saudara – saudara saya yang telah memberikan semangat, serta teman-teman
membantu dalam pengerjaan proposal ini. Ucapan terima kasih juga diperuntukan
kepada Dosen-dosen pembimbing proposal Tugas Akhir Telkom
University yang memberikan masukan dan saran terhadap proposal ini.

\vspace*{0.5cm}
\noindent Proposal penelitian ini bertujuan untuk mengembangkan ilmu teknologi
serta keamanan dalam bidang System on a Chip (SoC) yang masih jarang
dikembangkan di Indonesia. 

\vspace*{0.1cm}
\begin{flushright}
Bandung, 1 Juli 2017\\[0.1cm]
\vspace*{1cm}
\penulis

\end{flushright}

\tableofcontents

\clearpage
\listoffigures
\clearpage
\listoftables
\clearpage

\addChapter{DAFTAR SINGKATAN}
\chapter*{DAFTAR SINGKATAN}

\addChapter{DAFTAR SIMBOL}
\chapter*{DAFTAR SIMBOL}

\addChapter{DAFTAR ISTILAH}
\chapter*{DAFTAR ISTILAH}

\addChapter{DAFTAR LAMPIRAN}
\chapter*{DAFTAR LAMPIRAN}

\pagenumbering{arabic}

%%%%%%%%%%%%%%%%%%%%%%%%%%%%%%%%%%%%%%%%%%%%%%%%%%%%%%%%%%%
% 
%%%%%%%%%%%%%%%%%%%%%%%%%%%%%%%%%%%%%%%%%%%%%%%%%%%%%%%%%%%

\chapter{\babSatu}
Membuat desain IC membutuhkan sumber daya yang sangat banyak, serta prosedur dan ketelitian yang tinggi. oleh karena itu dalam prosesnya dibutuhkan pengamanan agar desain tidak mudah dicuri yang akan menimbulkan kerugian bagi produsen IC tersebut.

%%%%%%%%%%%%%%%%%%%%%%%%%%%%%%%%%%%%%%%%%%%%%%%%%%%%%%%%%%%
% 
%%%%%%%%%%%%%%%%%%%%%%%%%%%%%%%%%%%%%%%%%%%%%%%%%%%%%%%%%%%

\section{Latar Belakang}
\textit{Integrated Circuit} (IC) merupakan modul teknologi dasar dari perangkat elektronika tertanam modern. Dengan berkembangnya teknologi IC yang mengutamakan ukuran kecil, dan performa yang tinggi serta dengan harga yang murah membuat teknologi IC semakin diminati \cite{latex.intro}.

Dengan ukuran modul yang sangat kecil dan banyaknya komponen pembangun, kerja sama antara desainer dilakukan untuk membangun sebuah modul VLSI sehingga setiap desainer dapat fokus mendesain salah satu fungsi yang terdapat dalam modul tersebut. Kerja sama dilakukan untuk mempermudah pembuatan desain VLSI yang memiliki tingkat kerumitan yang tinggi. Desainer juga dapat mempercepat waktu mendesain dengan menggunakan kode sumber yang sudah ada atau bekerja sama secara \textit{paralel} membuat masing-masing modul yang nantinya akan digabung menjadi sebuah modul utama VLSI.

Setelah modul selesai dibuat maka modul siap untuk di-produksi. Dalam proses produksi modul perusahaan tempat desainer bekerja tidak perlu memiliki pabrik produksi modul sendiri, perusahaan dapat bekerja sama dengan mitra percetakan yang akan memproduksi modul buatan perusahaan modul tersebut. Cara kerja sama seperti ini disebut dengan \textit{Fabless Manufacturing} \cite{vlsi.hist}. Ketika akan memproduksi IC, perusahaan harus menyerahkan \textit{blueprint} modul VLSI ke percetakan, namun \textit{blueprint} tersebut tidak terjamin kerahasiaan nya serta memungkinkan plagiarisme desain oleh oknum perusahaan atau pihak ketiga yang tertarik menggunakan desain VLSI yang telah diserahkan untuk diproduksi ulang.

Dengan memberikan rangkaian \textit{watermark}/pelindung sebagai pengamanan pada \textit{blueprint} VLSI siap cetak yang menandakan kepemilikan dari desainer atau perusahaan produsen modul akan melindungi dari kecurangan pihak lain yang akan mencuri desain. Sehingga kemungkinan pencurian atau plagiarisme yang menyebabkan kerugian pada perusahaan atau desainer karena desain nya dicuri atau di-plagiat berkurang

%%%%%%%%%%%%%%%%%%%%%%%%%%%%%%%%%%%%%%%%%%%%%%%%%%%%%%%%%%%%
% 
%%%%%%%%%%%%%%%%%%%%%%%%%%%%%%%%%%%%%%%%%%%%%%%%%%%%%%%%%%%%

\section{Permasalahan}
Pada bagian ini akan dijelaskan mengenai definisi permasalahan yang dihadapi yang telah diselesaikan serta asumsi dan batasan yang digunakan dalam menyelesaikannya. Berikut ini dijelaskan rumusan masalah yang dihadapi dalam penelitian Intelectual Property Protection (IPP) menggunakan metode \textit{Digital Filter Obfuscation Algorithm} :


\subsection{Rumusan Masalahan}
Berikut ini dijelaskan rumusan masalah yang dihadapi dalam penelitian Intelectual Property Protection (IPP) menggunakan metode \textit{Digital Filter Obfuscation Algorithm} :

\begin{enumerate}
	\item Dengan metode \textit{Fabless Manufacturing}, desain modul yang siap diproduksi diserahkan kepada perusahaan percetakan mitra sehingga mitra dapat mengetahui desain modul dari desainer yang	memungkinkan desain dapat dicuri oleh oknum percetakan atau pihak	ketiga yang tertarik dengan desain tersebut.
	
	\item Desain modul rawan terhadap plagiarisme karena desain elektronik sangat mudah ditiru, sehingga pengamanan desain harus dilakukan agar desain tidak mudah untuk dicuri atau di-plagiat.
	
	\item Apabila pihak ketiga mencuri desain, desainer dapat mengklaim modul tersebut dengan bukti dari pengamanan watermark yang telah tertanam dalam IC menggunakan teknik pemanggilan watermark yang hanya diketahui oleh desainer yang mendesain IC tersebut.
\end{enumerate}

\subsection{Batasan Permasalahan}
Dalam penelitian ini rancangan desain VLSI yang disisipkan watermark membatasi masalah serta pembahasan yang akan diteliti sebagai berikut :

\begin{enumerate}
	\item Tidak membuat modul IC VLSI spesifik, namun menggunakan desain yang sudah ada kemudian disisipkan dengan rangkaian pelindung.
	
	\item Menyisipkan rangkaian dengan data pelindung dan tidak membahas data detail dari pemilik cipta.
	
	\item Watermarking yang dilakukan untuk satu chip IC dan tidak mewatermark masing-masing modul yang ter-integrasi dalam chip IC. 
\end{enumerate}

%%%%%%%%%%%%%%%%%%%%%%%%%%%%%%%%%%%%%%%%%%%%%%%%%%%%%%%%%%%%
% 
%%%%%%%%%%%%%%%%%%%%%%%%%%%%%%%%%%%%%%%%%%%%%%%%%%%%%%%%%%%%

\section{Tujuan}
Berikut merupakan tujuan pengamanan desain modul yang siap cetak sehingga aman terhadap pencurian hak cipta :
\begin{enumerate}
	\item Merancang rangkaian pengamanan dalam sebuah chip design sebagai bukti kepemilikan desain (\textit{ownership}) atau \textit{watermarking}.
	
	\item Desain \textit{chip} yang telah diberi rangkaian watermark akan dianalisis perubahan performa dari desain sebelum dan sesudah watermarking serta kemungkinan watermark di-modifikasi oleh pihak lain atau reverse engineering untuk digunakan kembali oleh pengguna yang tidak	sah.
	
	\item Rangkaian pengaman akan ditanam di dalam \textit{chip}/rangkaian utama yang jika di aktifkan akan memanggilan informasi pemilik dari \textit{chip} yang caranya hanya diketahui oleh pemilik cipta.
\end{enumerate}

%%%%%%%%%%%%%%%%%%%%%%%%%%%%%%%%%%%%%%%%%%%%%%%%%%%%%%%%%%%%%
% 
%%%%%%%%%%%%%%%%%%%%%%%%%%%%%%%%%%%%%%%%%%%%%%%%%%%%%%%%%%%%%

\section{Metodologi Penelitian}
Metode penelitian yang digunakan adalah perancangan dan \textit{prototyping} dan percobaan untuk membuktikan hipotesis yang ada.

%%%%%%%%%%%%%%%%%%%%%%%%%%%%%%%%%%%%%%%%%%%%%%%%%%%%%%%%%%%%%
% 
%%%%%%%%%%%%%%%%%%%%%%%%%%%%%%%%%%%%%%%%%%%%%%%%%%%%%%%%%%%%%

\section{Sistematika Penulisan}
Sistematika penulisan laporan adalah sebagai berikut:

\begin{itemize}
	
	\item Bab 1 \babSatu \\
	Berisi tentang uraian latar belakang, rumusan serta batasan masalah dan gambaran umum tentang penelitian sebelumnya yang sudah ada.
	
	\item Bab 2 \babDua \\
	Menjelaskan penjelasan singkat tentang LSI dan proses pembuatannya serta kemungkinan serangan dan cara mengatasinya. Cara mengatasi dari penelitian sebelumnya yang sudah ada dan cara yang diajukan oleh penulis.
	
	\item Bab 3 \babTiga \\
	Penjelasan detail tentang desain yang di ajukan penulis untuk mengatasi pembajakan desain serta simulasi hasil dari perancangan desain yang diajukan.

	\item Bab 4 \babEmpat \\
	Hasil pengujian terhadap desain yang diajukan penulis serta analisis terhadap desain yang diajukan.
	
	\item Bab 7 \kesimpulan \\
	Kesimpulan yang didapat dari hasil pengujian desain yang diajukan dan saran untuk pengembangan riset dimasa mendatang.
\end{itemize}
%-----------------------------------------------------------------------------%
\chapter{\babDua}
%-----------------------------------------------------------------------------%

%-----------------------------------------------------------------------------%
\section{Pekerjaan Sebelumnya dan Keterkaitan}
%-----------------------------------------------------------------------------%
Secara garis besar teknik Intelectual Property Protection (IPP)
watermarking dapat diklasifikasikan menjadi 2 kelas yaitu Dynamic
Watermarking dan Static Watermarking. Dynamic Watermarking merupakan
watermark yang tidak dapat terdeteksi kecuali dengan menjalankan IP yang telah
di-watermark untuk mendeteksi sinyal yang dihasilkan, seperti digital signal
processing (DSP), atau finite state mechine (FSM) watermarking. Static
Watermarking merupakan watermark yang mengacu pada properti dari sebuah
desain, dan hanya bisa terdeteksi dengan cara statis yang berbeda, seperti jalur dan
penempatan watermarking [7].

Salah satu pengamanan lain adalah mengonversi fail simulasi dari fail.
RTL source code yang memungkinkan tidak mudah untuk di-reverse-engineering
oleh pihak ketiga, sehingga model tidak dapat dirubah dan digunakan kembali
dengan keperluan lain oleh pihak ketiga dan pengguna yang tidak bertanggung
jawab.[8][9]

Namun cara tersebut hanya melindungi dari sisi softwere yang melindungi
IP agar tidak di-salah-gunakan oleh pengguna pihak ketiga. Untuk pengamanan IP
yang digunakan dalam sharing project dan reusable project dapat digunakan
dengan pengamanan Digital Signal Processing cell yang memungkinkan integrasi
dalam sistem.

Dalam penelitian kali ini akan melakukan kombinasi dari proteksi IP
polimorph gate dengan algoritme filter digital. Menggunakan gabungan dari dua
teknik ini akan memberikan tambahan keamanan pada proteksi IP yang
kemungkinan tingkat over write watermark lebih kecil. Oleh karena itu dalam
penelitian ini mengajukan sebuah gabungan metode yang sudah ada untuk
meningkatkan kemampuan pengamanan dalam sebuah modul VLSI yang sudah
ada. Dengan menggabungkan polygate sebagai kunci kombinasi untuk
mengaktifkan modul filter digital yang akan digunakan sebagai watermark.

\section{Perancangan dan Implementasi Algoritme DSP untuk IPP}

Melakukan analisis terhadap masalah yang dikaji kemudian akan
dilakukan rancangan Intelectual Property Protection (IPP) dengan algoritme Filter
Digital yang dibangun meliputi rangkaian uji. Dari desain modul VLSI yang telah
ada akan diuji coba kan performa sebelum diberi watermark.

Dengan memberikan rangkaian watermark sebagai pengamanan pada
blueprint VLSI siap cetak yang menandakan kepemilikan dari desainer atau
perusahaan produsen modul akan melindungi dari kecurangan pihak lain yang
akan mencuri desain tersebut. Sehingga kemungkinan pencurian atau plagiarisme
berkurang yang menyebabkan kerugian pada perusahaan atau desainer karena
desain nya dicuri atau diplagiat.

Desain akan dirancang dengan kombinasi Low Pass Filter, High Pass
Filter, Band Pass Filter, dan Band Reject Filter. Kombinasi ini akan ditentukan
dan diaktifkan oleh polygate sebagai kunci pengaktifan kombinasi Filter digital.
Setelah Filter digital aktif maka kombinasi data akan melewati kombinasi filter
yang diaktifkan dari kombinasi polygate. Kemudian data hasil kombinasi proses
ini akan membentuk pola khusus yang menjadi data watermark dari desainer yang
mencirikan identitas desainer. Setelah diberikan watermark maka modul akan
diuji coba kan kembali performa nya. Bila terjadi penurunan performa maka akan
dilakukan perbaikan algoritma kemudian dilakukan diuji kembali performa nya.
Hingga didapat performa yang paling baik dari beberapa uji coba yang akan
dilakukan.

%-----------------------------------------------------------------------------%
\chapter{\babTiga}
%-----------------------------------------------------------------------------%
%\todo{tambahkan kata-kata pengantar bab 1 disini}

%-----------------------------------------------------------------------------%
\section{Studi Literatur}
%-----------------------------------------------------------------------------%

\section{Analisis}

Analisis dilakukan untuk mengkaji masalah yang ada, mendefinisikan
batasan dalam masalah, lalu mencari solusi dari masalah tersebut. Analisis juga
meliputi performa rancangan modul yang telah diberi watermark yang diuji coba
dalam board FPGA.

\section{Perancangan}

Pada tahap ini dilakukan pengkajian masalah serta pendefinisian batasan
masalah. Pencarian solusi atas masalah yang muncul juga dilakukan. Tahap ini
juga meliputi analisis penempatan dan penentuan jalur dalam pemasangan
rangkaian uji IP Protection pada VLSI.

\begin{figure}[h]
	\centering
	\includegraphics[scale=0.6]{images/flow}
	\caption{Simulation results for the network.}
	\label{fig_sim}
\end{figure}

\begin{figure}[h]
	\centering
	\includegraphics[scale=0.8]{images/box}
	\caption{Simulation results for the network.}
	\label{fig_sim}
\end{figure}

Penelitian kali ini akan melakukan simulasi desain controller HDMI
menggunakan FPGA. Desain controller ini akan di-tes performa-nya dengan
menjalankan fail multimedia. Controller akan disambung-kan ke connector
HDMI lalu menampilkan hasil keluaran multimedia pada monitor atau TV.
Kemudian di dalam controller HDMI ini akan disisipkan rangkaian watermark
dan akan dilakukan pengecekan performa-nya lagi untuk mengetahui terjadinya
penurunan performa karena watermark.

% Table generated by Excel2LaTeX from sheet 'Data encripted'
\begin{table}[htbp]
	\centering
	\caption{Add caption}
	\begin{tabular}{rlll}
		\cline{2-3}    \multicolumn{1}{r|}{} & \multicolumn{1}{c|}{Alphabet} & \multicolumn{1}{c|}{Biner} &  \bigstrut\\
		\cline{2-3}    \multicolumn{1}{r|}{} & \multicolumn{1}{c|}{H} & \multicolumn{1}{c|}{111} &  \bigstrut\\
		\cline{2-3}    \multicolumn{1}{r|}{} & \multicolumn{1}{c|}{F} & \multicolumn{1}{c|}{101} &  \bigstrut\\
		\cline{2-3}    \multicolumn{1}{r|}{} & \multicolumn{1}{c|}{G} & \multicolumn{1}{c|}{110} &  \bigstrut\\
		\cline{2-3}    \multicolumn{1}{r|}{} & \multicolumn{1}{c|}{C} & \multicolumn{1}{c|}{010} &  \bigstrut\\
		\cline{2-3}    \multicolumn{1}{r|}{} & \multicolumn{1}{c|}{D} & \multicolumn{1}{c|}{011} &  \bigstrut\\
		\cline{2-3}    \multicolumn{1}{r|}{} & \multicolumn{1}{c|}{E} & \multicolumn{1}{c|}{100} &  \bigstrut\\
		\cline{2-3}    \multicolumn{1}{r|}{} & \multicolumn{1}{c|}{C} & \multicolumn{1}{c|}{010} &  \bigstrut\\
		\cline{2-3}    \multicolumn{1}{r|}{} & \multicolumn{1}{c|}{A} & \multicolumn{1}{c|}{000} &  \bigstrut\\
		\cline{2-3}    \multicolumn{1}{r|}{} & \multicolumn{1}{c|}{H} & \multicolumn{1}{c|}{111} &  \bigstrut\\
		\cline{2-3}    \multicolumn{1}{r|}{} & \multicolumn{1}{c|}{E} & \multicolumn{1}{c|}{100} &  \bigstrut\\
		\cline{2-3}    \multicolumn{1}{r|}{} & \multicolumn{1}{c|}{C} & \multicolumn{1}{c|}{010} &  \bigstrut\\
		\cline{2-3}    \multicolumn{1}{r|}{} & \multicolumn{1}{c|}{B} & \multicolumn{1}{c|}{001} &  \bigstrut\\
		\cline{2-3}    \multicolumn{1}{r|}{} & \multicolumn{1}{c|}{C} & \multicolumn{1}{c|}{010} &  \bigstrut\\
		\cline{2-3}    \multicolumn{1}{r|}{} & \multicolumn{1}{c|}{F} & \multicolumn{1}{c|}{101} &  \bigstrut\\
		\cline{2-3}    \multicolumn{1}{r|}{} & \multicolumn{1}{c|}{D} & \multicolumn{1}{c|}{011} &  \bigstrut\\
		\cline{2-3}    \multicolumn{1}{r|}{} & \multicolumn{1}{c|}{E} & \multicolumn{1}{c|}{100} &  \bigstrut\\
		\cline{2-3}    \multicolumn{1}{r|}{} & \multicolumn{1}{c|}{C} & \multicolumn{1}{c|}{010} &  \bigstrut\\
		\cline{2-3}    \multicolumn{1}{r|}{} & \multicolumn{1}{c|}{H} & \multicolumn{1}{c|}{111} &  \bigstrut\\
		\cline{2-3}    \multicolumn{1}{r|}{} & \multicolumn{1}{c|}{D} & \multicolumn{1}{c|}{011} &  \bigstrut\\
		\cline{2-3}          &       &       &  \bigstrut[t]\\
		INPUT : & \multicolumn{3}{l}{HFGCDECAHECBCFDECHD} \\
		OUTPUT : & \multicolumn{3}{l}{CABE} \\
	\end{tabular}%
	\label{tab:addlabel}%
\end{table}%



Dalam penelitian ini teknik watermark yang digunakan adalah
menggabungkan rangkaian logical polimorph gate sebagai kunci utama untuk
mengaktifkan rangkaian Digital Signal Filter yang akan diberi masukan kunci
kedua untuk memanggil data watermark dalam chip.

\begin{figure}[!h]
	\centering
	\includegraphics[scale=0.8]{images/polymorphgate}
	\caption{Simulation results for the network.}
	\label{fig_sim}
\end{figure}

Setelah unique key dari Polygates di masukan (contoh: K1 - K4), maka pin
input untuk algoritme DSP aktif (contoh: pin A – C) yang kemudian akan
mengolah key untuk penampilan watermark dalam chip.

Setelah polygate mengaktifkan filter yang dipilih maka data dari pin input
DSP (contoh: pin A – C) masuk ke dalam filter yang aktif dan akan diolah sebagai
data input watermark.

\section{Pengujian}

Pada tahap ini dilakukan serangkaikan uji coba untuk mengukur parameter
performa rangkaian VLSI yang telah disisipkan rangkaian uji IP Protection.
Petitcolas [13] mengidentifikasi beberapa hal yang menjadi bahan evaluasi untuk
IP Protection :

\subsection{Kerahasiaan algoritme}

Merujuk pada aturan keamanan yang dijelaskan oleh Kerckhoffs [14] pada
tahun 1883, setiap enkripsi atau teknik keamanan tidak boleh mengandalkan
kerahasiaan suatu algoritme, tetapi pada kompleksitas matematis algoritme
tersebut.

\subsection{Tingkat Ketahanan Uji}

Ini adalah aspek yang sangat penting. Aspek ini berisi tentang ketahanan
algoritme dari serangan dan persentase dari IP Protection tak terdeteksi.
Kemungkinan detector salah mendeteksi algoritme pada rangkaian tanpa
algoritme juga diperhitungkan di aspek ini.

\begin{figure}[!h]
	\centering
	\includegraphics[scale=0.8]{images/gate1}
	\caption{Simulation results for the network.}
	\label{fig_sim}
\end{figure}

\begin{figure}[!h]
	\centering
	\includegraphics[scale=0.8]{images/gatePlace}
	\caption{Simulation results for the network.}
	\label{fig_sim}
\end{figure}

\subsection{Tingkat Penurunan Performa}

Penurunan performa saat menyisipkan suatu metode IP Protection adalah
hal yang tak dapat dihindari. Tetapi penurunan performa yang terlalu besar akan
menjadi masalah. Maka dari itu, perbandingan performa antara rangkaian yang
telah disisipkan IP Protection dan rangkaian tanpa IP Protection harus dilakukan.

\subsection{Tingkat Deteksi}

Penyisipan watermark merupakan bagian dari proses, pelacak-kan dan
deteksi dalam teknik watermark IC. Pelacak-kan dan deteksi watermark pada
kemungkinan penyerangan merupakan aspek yang akan dijadikan pertimbangan
pada teknik watermark.

\section{Data Pengujian}

Metode yang akan digunakan merupakan simulasi pada board FPGA
dengan desain modul yang sudah ada dan menyisipkan suatu rangkaian tambahan
watermarking dan menguji perubahan performa modul yang telah di sisipkan
watermark tersebut.

Dalam penelitian ini akan menggunakan teknik Digital Signal Processing
Watermarking pada modul yang telah ada dengan kombinasi perhitungan loop
biner dengan output yang akan membentuk nama dari produsen asli modul
tersebut. Modul yang telah diberi watermark akan tetap dapat diuji keabsahan
pemiliknya walaupun modul telah digabungkan dengan modul lain dalam sebuah
proyek modul VLSI.

Kemudian setelah modul disisipkan watermark, kami akan menguji
performa modul tersebut dengan mengharapkan tidak ada perubahan berarti
terhadap modul yang telah diberi watermark tersebut.

%%%%%%%%%%%%%%%%%%%%%%%%%%%%%%%%%%%%%%%%%%%%%%%%%%%%%%%%%%
% 
%%%%%%%%%%%%%%%%%%%%%%%%%%%%%%%%%%%%%%%%%%%%%%%%%%%%%%%%%%

\chapter{\babEmpat}

%%%%%%%%%%%%%%%%%%%%%%%%%%%%%%%%%%%%%%%%%%%%%%%%%%%%%%%%%%
% 
%%%%%%%%%%%%%%%%%%%%%%%%%%%%%%%%%%%%%%%%%%%%%%%%%%%%%%%%%%

\section{Pengujian}
Setelah desain selesai di rancang, kemudian dilakukan pengujian untuk mengetahui apakah desain yang telah dirancang dapat bekerja dengan baik serta mengetahui bagaimana performasi alat yang di rancang.

\subsection{Sekenario Pengujian}
Untuk pengujian yang dilakukan adalah pengujian funsional algoritma (behavioral) serta pengujian performansi waktu operasi dan daya yang dibutuhkan alat yang telah dirancang.

Pada pengujian fungsional dilakukan pengujian mengguankan simulasi diagram waktu. Pada diagram waktu dapat dilihat proses input output satu demi satu dalam satuan waktu tertentu. Pengujian satu persatu dilakukan untuk mengetahui secara detail fungsional IC yang telah dirancang. Hasil pengujian fungsional dapat dilihat pada \textbf{lampiran 3}, hasil pengujian fungsional ini merujuk pada spesifikasi desain pada \textbf{lampiran 1 dan 2}.

Apabila uji fungsional tidak ditemukan anomali, maka dilanjutkan pengujian selanjutnya yaitu pengujian daya serta performansi. Hal ini untuk mengetahui apakah terjadi perubahan yang signifikan pada perangkat yang diberi perlindungan dengan yang tidak diberi perlindungan

\subsection{Hasil Pengujian}
Berikut ini adalah hasil pengujian fungsional dari perangkat yang telah dilindungi. pengujian yang pertama adalah pengujian fungsional  modul inti yaitu modul ALU.

\begin{table}[H]
	\centering
	\caption{Data Aktivasi rangkaian pelindung, Input (A) dan Output (Z)}
	\label{tab:rawdata}%
	\begin{tabular}{|c|c|c|c|c|c|c|c|c|c|c|c|c|c|c|c|c|c|}
		\hline
		\rowcolor[rgb]{ .949,  .949,  .949} \multicolumn{18}{|c|}{REG} \bigstrut\\
		\hline
		\rowcolor[rgb]{ .949,  .949,  .949} \multicolumn{1}{|p{2.145em}|}{CLK} & \multicolumn{8}{c|}{A}                                        &       & \multicolumn{8}{c|}{Z} \bigstrut\\
		\hline
		0     & 0     & 0     & 0     & 0     & 0     & 1     & 1     & 1     & \multirow{19}[38]{*}{} & 0     & 0     & 0     & 0     & 0     & 0     & 0     & 0 \bigstrut\\
		\cline{1-9}\cline{11-18}    1     & 0     & 0     & 0     & 0     & 0     & 1     & 0     & 1     &       & 0     & 1     & 0     & 0     & 0     & 0     & 0     & 0 \bigstrut\\
		\cline{1-9}\cline{11-18}    2     & 0     & 0     & 0     & 0     & 0     & 1     & 1     & 0     &       & 1     & 0     & 0     & 0     & 0     & 0     & 0     & 0 \bigstrut\\
		\cline{1-9}\cline{11-18}    3     & 0     & 0     & 0     & 0     & 0     & 0     & 1     & 0     &       & 1     & 1     & 0     & 0     & 1     & 0     & 1     & 0 \bigstrut\\
		\cline{1-9}\cline{11-18}    4     & 0     & 0     & 0     & 0     & 0     & 0     & 1     & 1     &       & 0     & 0     & 0     & 0     & 1     & 0     & 1     & 0 \bigstrut\\
		\cline{1-9}\cline{11-18}    5     & 0     & 0     & 0     & 0     & 0     & 1     & 0     & 0     &       & 0     & 1     & 0     & 0     & 1     & 0     & 1     & 0 \bigstrut\\
		\cline{1-9}\cline{11-18}    6     & 0     & 0     & 0     & 0     & 0     & 0     & 1     & 0     &       & 1     & 0     & 0     & 0     & 1     & 0     & 1     & 0 \bigstrut\\
		\cline{1-9}\cline{11-18}    7     & 0     & 0     & 0     & 0     & 0     & 0     & 0     & 0     &       & 1     & 1     & 0     & 1     & 0     & 0     & 0     & 0 \bigstrut\\
		\cline{1-9}\cline{11-18}    8     & 0     & 0     & 0     & 0     & 0     & 1     & 1     & 1     &       & 0     & 0     & 0     & 1     & 0     & 0     & 0     & 0 \bigstrut\\
		\cline{1-9}\cline{11-18}    9     & 0     & 0     & 0     & 0     & 0     & 1     & 0     & 0     &       & 0     & 1     & 0     & 1     & 0     & 0     & 0     & 0 \bigstrut\\
		\cline{1-9}\cline{11-18}    10    & 0     & 0     & 0     & 0     & 0     & 0     & 1     & 0     &       & 1     & 0     & 0     & 1     & 0     & 0     & 0     & 0 \bigstrut\\
		\cline{1-9}\cline{11-18}    11    & 0     & 0     & 0     & 0     & 0     & 0     & 0     & 1     &       & 1     & 1     & 0     & 1     & 1     & 0     & 0     & 1 \bigstrut\\
		\cline{1-9}\cline{11-18}    12    & 0     & 0     & 0     & 0     & 0     & 1     & 0     & 1     &       & 0     & 0     & 0     & 1     & 1     & 0     & 0     & 1 \bigstrut\\
		\cline{1-9}\cline{11-18}    13    & 0     & 0     & 0     & 0     & 0     & 0     & 1     & 1     &       & 0     & 1     & 0     & 1     & 1     & 0     & 0     & 1 \bigstrut\\
		\cline{1-9}\cline{11-18}    14    & 0     & 0     & 0     & 0     & 0     & 0     & 1     & 1     &       & 1     & 0     & 0     & 1     & 1     & 0     & 0     & 1 \bigstrut\\
		\cline{1-9}\cline{11-18}    15    & 0     & 0     & 0     & 0     & 0     & 1     & 0     & 0     &       & 1     & 1     & 1     & 0     & 0     & 1     & 0     & 0 \bigstrut\\
		\cline{1-9}\cline{11-18}    16    & 0     & 0     & 0     & 0     & 0     & 0     & 1     & 0     &       & 0     & 0     & 1     & 0     & 0     & 1     & 0     & 0 \bigstrut\\
		\cline{1-9}\cline{11-18}    17    & 0     & 0     & 0     & 0     & 0     & 1     & 1     & 1     &       & 0     & 1     & 1     & 0     & 0     & 1     & 0     & 0 \bigstrut\\
		\cline{1-9}\cline{11-18}    18    & 0     & 0     & 0     & 0     & 0     & 0     & 1     & 1     &       & 1     & 0     & 1     & 0     & 0     & 1     & 0     & 0 \bigstrut\\
		\hline
	\end{tabular}%
\end{table}%

Data mentah (raw) di atas didapat dari hasil timing diagram (lampiran). Untuk mengetahui hasil kebenaran dilakukan truth prove pada data raw. Signal dapat dilihat pada \textbf{lampiran B2}.

\begin{figure}
	\centering
	\includegraphics[width=0.8\textwidth]
	{pics/din.png}
	\caption{Diagram Sinyal Input}
	\label{Diagram Data Input}
\end{figure}

\begin{figure}
	\centering
	\includegraphics[width=0.8\textwidth]
	{pics/dout.png}
	\caption{Diagram Sinyal Output}
	\label{Diagram Data Output}
\end{figure}

Apabila digambarkan sinyal data input dan output untuk aktifasi perlindungan akan nampak seperti gambar diatas. Sinyal input seakan seperti sinyal tak beraturan namun memiliki pola tertentu. Jika kita analogikan seperti kunci dan gembok, sinyal input merupakan seakan seperti gerigi pada kunci yang akan mengurutkan susunan pada gembok agar terbuka.

Untuk mengecek apakah kunci merupakan kunci yang benar, maka dilakukan perhitungan autentikasi yang hasilnya seperti tabel dibawah ini.

\begin{table}[H]
	\centering
	\caption{Analisis Data Mentah}
	\label{tab:translasi}%
	\begin{tabular}{|c|c|c|c|c|c|c|c|c|c|c|c|c|}
		\hline
		\rowcolor[rgb]{ .949,  .949,  .949} \multicolumn{2}{|c|}{IO} & \multicolumn{8}{c|}{bit}                                      & \multicolumn{3}{c|}{STEP} \bigstrut\\
		\hline
		\rowcolor[rgb]{ .949,  .949,  .949} in    & out   & 7     & 6     & 5     & 4     & 3     & 2     & 1     & 0     & HCUT  & MCUT  & \multicolumn{1}{l|}{CHECK} \bigstrut\\
		\hline
		7     & 0     & 128   & 64    & 32    & 16    & 8     & 4     & 2     & 1     & 0     & 0     & - \bigstrut\\
		\hline
		5     & 64    & 128   & 64    & 32    & 16    & 8     & 4     & 2     & 1     & 0     & 0     & - \bigstrut\\
		\hline
		6     & 128   & 128   & 64    & 32    & 16    & 8     & 4     & 2     & 1     & 0     & 0     & - \bigstrut\\
		\hline
		2     & 202   & 128   & 64    & 32    & 16    & 8     & 4     & 2     & 1     & 10    & 2     & OK \bigstrut\\
		\hline
		3     & 10    & 128   & 64    & 32    & 16    & 8     & 4     & 2     & 1     & 0     & 0     & - \bigstrut\\
		\hline
		4     & 74    & 128   & 64    & 32    & 16    & 8     & 4     & 2     & 1     & 0     & 0     & - \bigstrut\\
		\hline
		2     & 138   & 128   & 64    & 32    & 16    & 8     & 4     & 2     & 1     & 0     & 0     & - \bigstrut\\
		\hline
		0     & 208   & 128   & 64    & 32    & 16    & 8     & 4     & 2     & 1     & 16    & 0     & OK \bigstrut\\
		\hline
		7     & 16    & 128   & 64    & 32    & 16    & 8     & 4     & 2     & 1     & 0     & 0     & - \bigstrut\\
		\hline
		4     & 80    & 128   & 64    & 32    & 16    & 8     & 4     & 2     & 1     & 0     & 0     & - \bigstrut\\
		\hline
		2     & 144   & 128   & 64    & 32    & 16    & 8     & 4     & 2     & 1     & 0     & 0     & - \bigstrut\\
		\hline
		1     & 217   & 128   & 64    & 32    & 16    & 8     & 4     & 2     & 1     & 25    & 1     & OK \bigstrut\\
		\hline
		5     & 25    & 128   & 64    & 32    & 16    & 8     & 4     & 2     & 1     & 0     & 0     & - \bigstrut\\
		\hline
		3     & 89    & 128   & 64    & 32    & 16    & 8     & 4     & 2     & 1     & 0     & 0     & - \bigstrut\\
		\hline
		3     & 153   & 128   & 64    & 32    & 16    & 8     & 4     & 2     & 1     & 0     & 0     & - \bigstrut\\
		\hline
		4     & 228   & 128   & 64    & 32    & 16    & 8     & 4     & 2     & 1     & 36    & 4     & OK \bigstrut\\
		\hline
		2     & 36    & 128   & 64    & 32    & 16    & 8     & 4     & 2     & 1     & 0     & 0     & - \bigstrut\\
		\hline
		7     & 100   & 128   & 64    & 32    & 16    & 8     & 4     & 2     & 1     & 0     & 0     & - \bigstrut\\
		\hline
		3     & 164   & 128   & 64    & 32    & 16    & 8     & 4     & 2     & 1     & 0     & 0     & - \bigstrut\\
		\hline
	\end{tabular}%
\end{table}%

Dari hasil pengujian di atas didapat bahwa fungsional dari masing masing modul yang telah digabung tidak saling bentrok satu sama lain dan dapat bekerja dengan semestinya.

Dengan hasil pengujian fungsional yang tidak terdapat anomali maka dapat dilanjutkan analisis daya dari modul sebelum diberi perlindungan dengan modul yang telah diberi perlindungan.

%%%%%%%%%%%%%%%%%%%%%%%%%%%%%%%%%%%%%%%%%%%%%%%%%%%%%%%%%%
% 
%%%%%%%%%%%%%%%%%%%%%%%%%%%%%%%%%%%%%%%%%%%%%%%%%%%%%%%%%%

\section{Analisis}
Pertama analisis performansi perangkat sebelum dilindungi, kemudian lindungi alat dengan rangkaian pelindung dan analisis performansinya dan bandingkan hasil analisis sebelum dengan hasil analisis sesudah dilindungi. berikut rekap data hasil analisis sebelum dan sesudah diberi rangkaian pelindung. Berikut merupakan soft-estimasi kecepatan clock pada FPGA arsitektur XILINX.

\begin{table}[H]
	\centering
	\caption{FPGA Speed Analysis}
	\label{tab:speed}%
	\begin{tabular}{|l|c|c|}
		\hline
		\rowcolor[rgb]{ .906,  .902,  .902} \multicolumn{1}{|c|}{Unprotected} & Minimum & Maximum \bigstrut\\
		\hline
		Period & 2.692ns & 371.471MHz (freq) \bigstrut\\
		\hline
		Input arrival time before clock & 10.075ns & - \bigstrut\\
		\hline
		Output required time after clock & -     & 5.558ns \bigstrut\\
		\hline
		\rowcolor[rgb]{ .906,  .902,  .902} \multicolumn{1}{|c|}{Protected} & Minimum & Maximum \bigstrut\\
		\hline
		Period & 4.023ns & 248.571MHz (freq) \bigstrut\\
		\hline
		Input arrival time before clock & 8.667ns & - \bigstrut\\
		\hline
		Output required time after clock & -     & 6.962ns \bigstrut\\
		\hline
	\end{tabular}%
\end{table}%

\begin{figure}
	\centering
	\includegraphics[width=1.0\textwidth]
	{pics/kecepatan.png}
	\caption{Power Supply Currents Diagram}
	\label{DiagramKecepatan}
\end{figure}

Dari hasil analisis terjadi penurunan kecepatan maksimum proses pada FPGA dari 371.471 Mhz menjadi 248.571 Mhz atau sekitar 33\%. Pada hasil soft-simulasi ini mengindikasikan bakal terjadi penurunan speed pada modul yang sedang dikembangkan.

\begin{table}[H]
	\centering
	\caption{On-Chip Power Summary}
	\label{tab:power}%
	\begin{tabular}{|l|c|}
		\hline
		\rowcolor[rgb]{ .906,  .902,  .902} \multicolumn{1}{|c|}{Unprotected} & Power (mW) \bigstrut\\
		\hline
		Clock & 1.12 \bigstrut\\
		\hline
		Static Power & 10.42 \bigstrut\\
		\hline
		Total & 11.54 \bigstrut\\
		\hline
		\rowcolor[rgb]{ .906,  .902,  .902} \multicolumn{1}{|c|}{Protected} & Power (mW) \bigstrut\\
		\hline
		Clock & 1.37 \bigstrut\\
		\hline
		Static Power & 10.42 \bigstrut\\
		\hline
		Total & 11.79 \bigstrut\\
		\hline
	\end{tabular}%
\end{table}%

\begin{figure}
	\centering
	\includegraphics[width=1.0\textwidth]
	{pics/daya.png}
	\caption{On-Chip Power Summary Diagram}
	\label{DiagramDaya}
\end{figure}

Dari data tersebut, kebutuhan daya total rangkaian utama sesudah proteksi meningkat sebesar 2\%, dari 11.54mW menjadi 11.79mW.

\begin{table}[H]
	\centering
	\caption{Power Supply Currents}
	\label{tab:addlabel}%
	\begin{tabular}{|c|c|c|c|c|}
		\hline
		\rowcolor[rgb]{ .906,  .902,  .902} \multicolumn{5}{|c|}{Unprotected} \bigstrut\\
		\hline
		\rowcolor[rgb]{ .906,  .902,  .902} \multicolumn{1}{|p{4.93em}|}{Supply Source} & \multicolumn{1}{p{4.93em}|}{Supply Voltage} & \multicolumn{1}{p{4.93em}|}{Total Current (mA)} & \multicolumn{1}{p{4.93em}|}{Dynamic Current (mA)} & \multicolumn{1}{p{4.93em}|}{Quiescent Current (mA)} \bigstrut\\
		\hline
		Vccint & 1.20  & 2.95  & 0.94  & 2.01 \bigstrut\\
		\hline
		Vccaux & 2.50  & 3.00  & 0.00  & 3.00 \bigstrut\\
		\hline
		Vcco25 & 2.50  & 0.20  & 0.00  & 0.20 \bigstrut\\
		\hline
		\rowcolor[rgb]{ .906,  .902,  .902} \multicolumn{5}{|c|}{Protected} \bigstrut\\
		\hline
		\rowcolor[rgb]{ .906,  .902,  .902} \multicolumn{1}{|p{4.93em}|}{Supply Source} & \multicolumn{1}{p{4.93em}|}{Supply Voltage} & \multicolumn{1}{p{4.93em}|}{Total Current (mA)} & \multicolumn{1}{p{4.93em}|}{Dynamic Current (mA)} & \multicolumn{1}{p{4.93em}|}{Quiescent Current (mA)} \bigstrut\\
		\hline
		Vccint & 1.20  & 3.16  & 1.14  & 2.02 \bigstrut\\
		\hline
		Vccaux & 2.50  & 3.00  & 0.00  & 3.00 \bigstrut\\
		\hline
		Vcco25 & 2.50  & 2.00  & 0.00  & 0.20 \bigstrut\\
		\hline
	\end{tabular}%
\end{table}%

\begin{figure}
	\centering
	\includegraphics[width=1.0\textwidth]
	{pics/powerConsumption.png}
	\caption{Power Supply Currents Diagram}
	\label{Power Consumption}
\end{figure}


Dari hasil diatas terlihat kebutuhan arus pada FPGA meningkat dari 2.95 menjadi 3.16 atau sekitar 6.64\% pada suplay Vccint di 1.2 Volt.


% Table generated by Excel2LaTeX from sheet 'Sheet2'
\begin{table}[htbp]
  \centering
  \caption{Peningkatan overhead yang digunakan setelah kompilasi}
    \begin{tabular}{lllllllll}
    \hline
    \multicolumn{3}{|c|}{} & \multicolumn{2}{c|}{\cellcolor[rgb]{ .851,  .851,  .851}unprotected} & \multicolumn{2}{c|}{\cellcolor[rgb]{ .851,  .851,  .851}protected} & \multicolumn{2}{c|}{\cellcolor[rgb]{ .851,  .851,  .851}protector} \bigstrut\\
    \hline
    \multicolumn{1}{c|}{\multirow{7}[14]{*}{used}} & \multicolumn{1}{c|}{Gate} & \multicolumn{1}{c|}{size} & \multicolumn{1}{c|}{\cellcolor[rgb]{ .851,  .851,  .851}gates} & \multicolumn{1}{c|}{tspg} & \multicolumn{1}{c|}{\cellcolor[rgb]{ .851,  .851,  .851}gates} & \multicolumn{1}{c|}{tspg} & \multicolumn{1}{c|}{\cellcolor[rgb]{ .851,  .851,  .851}gates} & \multicolumn{1}{c|}{tspg} \bigstrut\\
\cline{2-9}    \multicolumn{1}{c|}{} & \multicolumn{1}{c|}{not} & \multicolumn{1}{c|}{6} & \multicolumn{1}{c|}{\cellcolor[rgb]{ .851,  .851,  .851}365} & \multicolumn{1}{c|}{2190} & \multicolumn{1}{c|}{\cellcolor[rgb]{ .851,  .851,  .851}376} & \multicolumn{1}{c|}{2256} & \multicolumn{1}{c|}{\cellcolor[rgb]{ .851,  .851,  .851}7} & \multicolumn{1}{c|}{42} \bigstrut\\
\cline{2-9}    \multicolumn{1}{c|}{} & \multicolumn{1}{c|}{nand21} & \multicolumn{1}{c|}{4} & \multicolumn{1}{c|}{\cellcolor[rgb]{ .851,  .851,  .851}615} & \multicolumn{1}{c|}{2460} & \multicolumn{1}{c|}{\cellcolor[rgb]{ .851,  .851,  .851}634} & \multicolumn{1}{c|}{2536} & \multicolumn{1}{c|}{\cellcolor[rgb]{ .851,  .851,  .851}2} & \multicolumn{1}{c|}{8} \bigstrut\\
\cline{2-9}    \multicolumn{1}{c|}{} & \multicolumn{1}{c|}{nor21} & \multicolumn{1}{c|}{4} & \multicolumn{1}{c|}{\cellcolor[rgb]{ .851,  .851,  .851}1117} & \multicolumn{1}{c|}{4468} & \multicolumn{1}{c|}{\cellcolor[rgb]{ .851,  .851,  .851}1116} & \multicolumn{1}{c|}{4464} & \multicolumn{1}{c|}{\cellcolor[rgb]{ .851,  .851,  .851}12} & \multicolumn{1}{c|}{48} \bigstrut\\
\cline{2-9}    \multicolumn{1}{c|}{} & \multicolumn{1}{c|}{xor21} & \multicolumn{1}{c|}{8} & \multicolumn{1}{c|}{\cellcolor[rgb]{ .851,  .851,  .851}438} & \multicolumn{1}{c|}{3504} & \multicolumn{1}{c|}{\cellcolor[rgb]{ .851,  .851,  .851}419} & \multicolumn{1}{c|}{3352} & \multicolumn{1}{c|}{\cellcolor[rgb]{ .851,  .851,  .851}1} & \multicolumn{1}{c|}{8} \bigstrut\\
\cline{2-9}    \multicolumn{1}{c|}{} & \multicolumn{1}{c|}{dff} & \multicolumn{1}{c|}{18} & \multicolumn{1}{c|}{\cellcolor[rgb]{ .851,  .851,  .851}16} & \multicolumn{1}{c|}{288} & \multicolumn{1}{c|}{\cellcolor[rgb]{ .851,  .851,  .851}16} & \multicolumn{1}{c|}{288} & \multicolumn{1}{c|}{\cellcolor[rgb]{ .851,  .851,  .851}16} & \multicolumn{1}{c|}{288} \bigstrut\\
\cline{2-9}    \multicolumn{1}{c|}{} & \multicolumn{1}{c|}{mux21} & \multicolumn{1}{c|}{6} & \multicolumn{1}{c|}{\cellcolor[rgb]{ .851,  .851,  .851}126} & \multicolumn{1}{c|}{756} & \multicolumn{1}{c|}{\cellcolor[rgb]{ .851,  .851,  .851}129} & \multicolumn{1}{c|}{774} & \multicolumn{1}{c|}{\cellcolor[rgb]{ .851,  .851,  .851}5} & \multicolumn{1}{c|}{30} \bigstrut\\
    \hline
    \multicolumn{3}{|c|}{total} & \multicolumn{1}{c|}{2677} & \multicolumn{1}{c|}{13666} & \multicolumn{1}{c|}{2690} & \multicolumn{1}{c|}{13670} & \multicolumn{1}{c|}{43} & \multicolumn{1}{c|}{424} \bigstrut\\
    \hline
    \multicolumn{3}{|c|}{differential area size} & \multicolumn{4}{c|}{0.03\%}   & \multicolumn{2}{c|}{\multirow{2}[4]{*}{}} \bigstrut\\
\cline{1-7}    \multicolumn{3}{|c|}{differential gate used} & \multicolumn{4}{c|}{0.49\%}   & \multicolumn{2}{c|}{} \bigstrut\\
    \hline
    \multicolumn{9}{c}{} \bigstrut[t]\\
    \multicolumn{9}{l}{* public standard cell size (square microns)} \\
    \multicolumn{9}{l}{** tspg (total size per gate)} \\
    \end{tabular}%
  \label{tabaddlabel}%
\end{table}%


Kompilasi menggunakan library standard dari mosis didapatkan banyaknya gate yang digunakan sebelum rangkaian dilindungi adalah 2677 dan meningkat menjadi 2690 setelah rangkaian diberi pelindung. Peningkatan jumlah gate yang digunakan meningkat sekitar 0.03\%. Serta dari Luas 13666 micron persegi menja 13679 micron persegi, dengan kata lain terjadi penambahan overhead sekitar 0.49\%. Artinya peningkatan luas layoutnya tidak terlalu signifikan bila di sisipkan rangkaian pelindung.
%%%%%%%%%%%%%%%%%%%%%%%%%%%%%%%%%%%%%%%%%%%%%%%%%%%%%%%%%%%%%
% 
%%%%%%%%%%%%%%%%%%%%%%%%%%%%%%%%%%%%%%%%%%%%%%%%%%%%%%%%%%%%%

\chapter{\kesimpulan}

%%%%%%%%%%%%%%%%%%%%%%%%%%%%%%%%%%%%%%%%%%%%%%%%%%%%%%%%%%%%%
% 
%%%%%%%%%%%%%%%%%%%%%%%%%%%%%%%%%%%%%%%%%%%%%%%%%%%%%%%%%%%%%

\section{Kesimpulan}
Meninjau hasil analisis yang di lakukan dari pengujian yang telah dilakukan, di dapat kesimpulan sebagai berikut.

\begin{enumerate}	
	\item Masih memungkinkan menyisipkan suatu rangkaian pelindung ke dalam rangkaian utama tanpa mengganggu fungsi utama rangkaian.
	
	\item Data signature dapat di panggil, sehingga desain dapat diklaim kepemilikannya. 
	
	\item Terjadi penurunan kecepatan proses serta peningkatan kebutuhan daya pada hasil analisis simulasi di fpga.
	
	\item Dari hasil kompilasi gerbang, komponen komponen yang digunakan untuk fungsi rangkaian utama dan fungsi rangkaian pelindung menyatu, sehingga bisa digunakan sebagai pengecoh agar rangkaian sulit ditiru.
\end{enumerate}

%%%%%%%%%%%%%%%%%%%%%%%%%%%%%%%%%%%%%%%%%%%%%%%%%%%%%%%%%%%%
% 
%%%%%%%%%%%%%%%%%%%%%%%%%%%%%%%%%%%%%%%%%%%%%%%%%%%%%%%%%%%%

\section{Saran}
Agar Teknologi pengamanan Intelektual Properti lebih maju serta memperbaiki permasalahan yang masih ada pada penelitian ini, berikut beberapa saran dari untuk pengembangan dan penelitian selanjutnya:

\begin{enumerate}
	
	\item Penelitian ini dilakukan pada layer software, untuk meningkatkan kecepatan akses dan mengurangi konsumsi daya dari hasil analisis level behavioral pada fpga, dibutuhkan analisis lebih lanjut pada syntesis level gate (netlist) dan level phisical (layout).
	
	\item Saat ini teknologi serta teknik perlindungan properti intelektual perangkat keras masih terbilang baru, bidang keamanan pada IC masih minim resource serta proses manufakturing IC sendiri begitu kompleks dan luas serta spesifikasi desain setiap produk sangat rahasia. Dibutuhkan kajian khusus keamanan pada Level fabrikasi seperti RTL Level, Gate Level dan Layout Level.
\end{enumerate}


%
% Daftar Pustaka 
% 

% 
% Tambahkan pustaka yang digunakan setelah perintah berikut. 
% 
\begin{thebibliography}{4}

\bibitem{latex.intro}
{Jeff Clark. (n.d). \f{Introduction to LaTeX}.
26 Januari 2010. \url{http://frodo.elon.edu/tutorial/tutorial/node3.html}.}

\bibitem{chapman}
R. Chapman and T. S. Durrani, “IP Protection of DSP Algorithms for System on Chip Implementation,” vol. 48, no. 3, pp. 854–861, 2000.

\bibitem{water}
“Watermarking Techniques for Electronic Circuit Design,” no. 1, pp. 1–17.

\bibitem{lui}
Q. Liu, W. Ji, Q. Chen, and T. Mak, “IP Protection of Mesh NoCs Using Square Spiral Routing,” vol. 24, no. 4, pp. 1560–1573, 2016.

\bibitem{cui}
A. Cui, C. Chang, S. Member, S. Tahar, and S. Member, “A Robust FSM Watermarking Scheme for IP Protection of Sequential Circuit Design,” vol. 30, no. 5, pp. 678–690, 2011.

\bibitem{nie}
T. Nie, “Performance Evaluation for IP Protection Watermarking Techniques.”

\bibitem{zhang1}
J. Zhang, Y. Lin, Y. Lyu, G. Qu, and S. Member, “A PUF-FSM Binding Scheme for FPGA IP Protection and Pay-Per-Device Licensing,” vol. 10, no. 6, pp. 1137–1150, 2015.

\bibitem{zhang2}
J. Zhang, Y. Lin, Q. Wu, and W. Che, “Watermarking FPGA Bitfile for Intellectual Property Protection,” pp. 764–771.

\bibitem{kahng}
A. B. Kahng et al., “Watermarking Techniques for Intellectual Property Protection.”

\bibitem{mosh}
V. G. Moshnyaga and H. Nita, “STG-based Detection of Power Virus Inputs in FSM.”

\end{thebibliography}



\begin{appendix}
	\include{markLampiran}
	\setcounter{page}{2}
	%%%%%%%%%%%%%%%%%%%%%%%%%%%%%%%%%%%%%%%%%%%%%%%%%%%%%%%%%%%%%
% 
%%%%%%%%%%%%%%%%%%%%%%%%%%%%%%%%%%%%%%%%%%%%%%%%%%%%%%%%%%%%%

%\addChapter{Lampiran A Datasheet}
\begin{titlepage}
	\centering 
	\vspace*{9cm}
	\noindent \Huge{Lampiran A \\ Datasheet}
	\addChapter{LAMPIRAN A}

\end{titlepage}
\chapter*{\vspace*{9cm} A1  Komersial Datasheet}
\pagenumbering{alph}
\putpdf{lampiran/VLSI1}
\chapter*{\vspace*{9cm} A2 Internal/Developer Datasheet}
\putpdf{lampiran/VLSI2}

\begin{titlepage}
	\centering 
	\vspace*{9cm}
	\noindent \Huge{Lampiran B \\ Test Bench}
	\addChapter{LAMPIRAN B}
\end{titlepage}
\chapter*{\vspace*{9cm} B1 ALU Test Bench - ALU Active}
\includepdf[pages={1,2}, angle=-90, scale=0.9]{lampiran/aluTest1}
\chapter*{\vspace*{9cm} B2 ALU Test Bench - Protection Active}
\includepdf[pages={1}, angle=-90, scale=0.9]{lampiran/aluTest2}

\begin{titlepage}
	\centering 
	\vspace*{9cm}
	\noindent \Huge{Lampiran C \\ RTL Design
	\addChapter{LAMPIRAN C}
\end{titlepage}

\chapter*{\vspace*{9cm} C1 RTL Top Module}
\includepdf[pages={1}, angle=-90, scale=0.9]{lampiran/rtlTop}
\chapter*{\vspace*{9cm} C2 RTL Protection}
\includepdf[pages={1}, angle=0, scale=0.9]{lampiran/rtlPro}

\end{appendix}

\end{document}